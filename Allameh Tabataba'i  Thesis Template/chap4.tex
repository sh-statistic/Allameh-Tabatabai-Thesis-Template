
\chapter{روش‌های هم‌محلی موضعی مبتنی بر توابع پایه شعاعی  }\label{se:rbfloc}
%
در فصل قبل، روش هم‌محلی توابع پایه شعاعی برای معادلات دیفرانسیل با مشتقات جزیی وابسته به زمان مورد مطالعه قرار گرفت. در این فصل، فرمولبندی دیگری از توابع پایه شعاعی که منجر به درونیابی موضعی با نقاط پراکنده  می‌شود، مورد بررسی قرار می‌گیرد. برخلاف روش‌های فراگیر، درونیابی موضعی منجر به ماتریس ضرایب تنک می‌شود. در این فصل، در حالت کلی دو روش موضعی به شرح زیر بررسی خواهد شد.

\begin{enumerate}
	\item توابع پایه شعاعی مبتنی بر تفاضلات متناهی
	\item روش توابع پایه شعاعی مبتنی افراز واحد
\end{enumerate}

\section{روش افراز واحد}