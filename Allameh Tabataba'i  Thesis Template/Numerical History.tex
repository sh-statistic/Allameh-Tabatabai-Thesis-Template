در این قسمت به طور خلاصه به تاریخچه توسعه روشهای عددی برای حل معادلات انتگرال ولترا می‌پردازیم. موضوع اولین مقاله‌های ارائه شده توسط آبل []، [] و ولترا [] در مورد وارون انتگرال‌های معین به این صورت است: اگر توابع $g$ و $G$ توابع معلوم به ترتیب یک و دو متغیره باشند، تابع $y$ را طوری پیدا کنید که معادله انتگرال نوع اول زیر برقرار باشد:
\begin{align}\label{eq:101}
int_0^t(t-s)^{-\alpha}G(t,s)y(s)ds=g(t),~~~~T\in I, 0\leq\alpha\<1.
\end{align}
آبل حالت خاص $G(s,t)=1$ و $0<\alpha<1$ را مورد بررسی قرار داد و فرمول صریح زیر را بدست آورد:
\begin{align}\label{eq:102}
y(t)=\dfrac{\sin(\alpha\pi)}{\pi}\dfrac{d}{dt}(\int_0^t(t-s)^{\alpha-1}g(s)ds).
\end{align}
او نشان داد که معادله انتگرال (\ref{eq:101}) با $G(s,t)=1$ مسئله تعیین معادله یک منحنی، طوری که زمان لازم برای لغزیدن یک نقطه جرمی روی آن، تحت تاثیر نیروی گرانش از یک ارتفاع مشخص تا روی محور افقی برابر با یک تابع معین از ارتفاع باشد.\\
حالت کلی معادله در سال 1896 توسط ولترا [] بررسی شد و نشان داد برای هر دو حالت $\alpha=0$ و $\alpha\in(0,1)$، اگر برای هر $t\in I$، $G(t,t)\neq0$ و اگر $g$ و $G$ در بعضی شرایط همواری صدق کنند، بنابراین معادله (\ref{eq:101}) را می‌توان به صورت زیر بازنویسی کرد:
\begin{align}\label{eq:103}
y(t)=.
\end{align}