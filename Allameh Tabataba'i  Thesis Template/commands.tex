% در این فایل، دستورها و تنظیمات مورد نیاز، آورده شده است.
%-------------------------------------------------------------------------------------------------------------------

% در ورژن جدید زی‌پرشین برای تایپ متن‌های ریاضی، این سه بسته، حتماً باید فراخوانی شود
\usepackage{amsthm,amssymb,amsmath}
% بسته‌ای برای تنطیم حاشیه‌های بالا، پایین، چپ و راست صفحه
\usepackage[top=40mm, bottom=35mm, left=25mm, right=30mm]{geometry}
% بسته‌‌ای برای ظاهر شدن شکل‌ها و تصاویر متن
\usepackage{graphicx}
\usepackage{xcolor}
%\usepackage{hyperref} 							% 	PDF links
\usepackage{setspace} 				    		% 	for switching between double/single space in document
\graphicspath{{images}}
\usepackage{verbatim}

\usepackage{natbib}                             %   برای ظاهر شدن مراجع فارسی درصورت استفاده از بیب فایل%
                                              
                    %  براي ارجاعات با پرانتز سطر فوق را فعال كرده و سطر زير را غير فعال كنيد.
%\usepackage[nonamebreak,square]{natbib}%nonamebreak,numbers,

%\usepackage{tocbibind}  % بسته‌ای برای ظاهر شدن «مراجع» و «نمایه» در فهرست مطالب

% بسته‌ای برای رسم کادر
\usepackage{framed} 
% بسته‌‌ای برای چاپ شدن خودکار تعداد صفحات در صفحه «معرفی پایان‌نامه»
\usepackage{lastpage}
% بسته‌‌ای برای ایجاد دیاگرام‌های مختلف
\usepackage[all]{xy}
% بسته‌ و دستوراتی برای ایجاد لینک‌های رنگی با امکان جهش
\usepackage[pdfmenubar=true,pagebackref=false,colorlinks=true,linkcolor=blue,citecolor=magenta,linktoc=none]{hyperref}
% چنانچه قصد پرینت گرفتن نوشته خود را دارید، خط بالا را غیرفعال و  از دستور زیر استفاده کنید چون در صورت استفاده از دستور زیر‌‌، 
% لینک‌ها به رنگ سیاه ظاهر خواهند شد که برای پرینت گرفتن، مناسب‌تر است
%\usepackage[pagebackref=false]{hyperref}
% بسته‌ لازم برای تنظیم سربرگ‌ها
\usepackage{fancyhdr}
% بسته‌ای برای ظاهر شدن «مراجع» و «نمایه» در فهرست مطالب
\usepackage[nottoc]{tocbibind}
% دستورات مربوط به ایجاد نمایه
\usepackage{makeidx}
\makeindex
 
\setcounter{tocdepth}{4}                                        %دستوری برای ایجاد زیر بخش‌های تودرتو همراه با شماره در فهرست
\setcounter{secnumdepth}{4}
\usepackage{textcomp}         %بسته‌ای برای غیر قابل کپی برداری شدن فایل پی دی اف
%%%%%%%%%%%%%%%%%%%%%%%%%%
 %                                                       ---------------------------------------------------------		
 	                                      %   فراخوانی بسته زی‌پرشین و تعریف قلم فارسی و انگلیسی %
 	                                                             %XePersian  دستورات مربوط به 
\usepackage{xepersian}
\settextfont[Scale=1.15]{XB Niloofar}
\setdigitfont{XB Niloofar}
\setlatintextfont{LinLibertine}
%\setdigitfont[Scale=1]{Times New Roman}			        
% 																	-------------------------------------
% از revision 118 زی‌پرشین به بعد، وارد کردن دستور زیر لازم نیست. توجه داشته باشید که در صورت  غیرفعال کردن این %دستور،
% از فونت پیش‌فرض لاتک برای کلمات انگلیسی استفاده خواهد شد.
\setlatintextfont[ExternalLocation,BoldFont={lmroman10-bold},BoldItalicFont={lmroman10-bolditalic},ItalicFont={lmroman10-italic}]{lmroman10-regular}
%%%%%%%%%%%%%%%%%%%%%%%%%%
% چنانچه می‌خواهید اعداد در فرمول‌ها، انگلیسی نباشد، خط زیر را غیرفعال کنید
\DefaultMathDigits
%%%%%%%%%%%%%%%%%%%%%%%%%%
% تعریف قلم‌های فارسی و انگلیسی اضافی برای استفاده در بعضی از قسمت‌های متن
\defpersianfont\nastaliq[Scale=2]{IranNastaliq}
\defpersianfont\chapternumber[Scale=1.5]{XB Niloofar}
\defpersianfont\titr[Scale=1]{XB Titre}
\settextfont{Persian Modern}
%%%%%%%%%%%%%%%%%%%%%%%%%%
%
%% دستوری برای حذف کلمه «چکیده»
%\renewcommand{\abstractname}{}
%% دستوری برای حذف کلمه «abstract»
%\renewcommand{\latinabstract}{}
% دستوری برای تغییر نام کلمه «اثبات» به «برهان»
\renewcommand\proofname{\textbf{برهان}}
% دستوری برای تغییر نام کلمه «کتاب‌نامه» به «مراجع»
\renewcommand{\bibname}{مرجع‌ها}
%دستوری برای اینکه لیست جدول و شکل به فهرست تبدیل شود:
\def\listfigurename{\if@RTL List of Figures \else فهرست‌ شکل‌‌ها \fi}
\def\listtablename{\if@RTL List of Tables ‌\else  فهرست جدول‌ها\fi}
% دستوری برای تعریف واژه‌نامه انگلیسی به فارسی
\newcommand\persiangloss[2]{#1\hfill\lr{#2}\\}
% دستوری برای تعریف واژه‌نامه فارسی به انگلیسی 
\newcommand\englishgloss[2]{#2\dotfill\lr{#1}\\}
%%%%%%%%%%%%%%%%%%%%%%%%%
\newcommand{\blockmatrix}[3]{%These end of the line comments are neccessary
\begin{minipage}[t][#2][c]{#1}%
\center%
#3%
\end{minipage}%
}%
\newcommand{\fblockmatrix}[3]{%
\fbox{%
\begin{minipage}[t][#2][c]{#1}%
\center%
#3%
\end{minipage}%
}%
}

%%%%%%%%%%%%%%%%%%%%%%%%%%
% تعریف و نحوه ظاهر شدن عنوان قضیه‌ها، تعریف‌ها، مثال‌ها و ...
\newtheorem{theo}{قضیه}[section]
\renewcommand{\thetheo}{\arabic{theo}.\arabic{section}.\arabic{chapter}}
\newcommand{\mored}[2]{{\large \theo{\bf #2.} \label{#1}}}
\newtheorem{lem}{لم}[section]
\renewcommand{\thelem}{\arabic{lem}.\arabic{section}.\arabic{chapter}}
%\newcommand{\mored}[2]{{\large \lem{\bf #2.} \label{#1}}}
\newtheorem{example}{مثال}[section]
%\renewcommand{\theexample}{\arabic{example}.\arabic{section}.\arabic{chapter}}
%\newcommand{\mored}[2]{{\large \example{\bf #2.} \label{#1}}}

\theoremstyle{definition}
\newtheorem{definition}{تعریف}[section]
%\theoremstyle{theorem}
%\newtheorem{theorem}[theo]{قضیه}
%\newtheorem{lemma}[theo]{لم}
\newtheorem{proposition}[theo]{گزاره}
\newtheorem{corollary}[definition]{نتیجه}
\newtheorem{remark}[definition]{ملاحظه}
%\theoremstyle{definition}
%\newtheorem{example}[definition]{مثال}
%%%%%%%%%%%%%%%%%%%%%%%%%%
% تعریف دستورات جدید برای خلاصه نویسی و راحتی کار در هنگام تایپ فرمول‌های ریاضی
\newcommand{\bR}{\mathbb{R}}
\newcommand{\cB}{\mathcal{B}}
\newcommand{\cO}{\mathcal{O}}
\newcommand{\cG}{\mathcal{G}}
\newcommand{\cU}{\mathcal{U}}
\newcommand{\cK}{\mathcal{K}}
\newcommand{\cS}{\mathcal{S}}
\newcommand{\rM}{\mathrm{M}}
\newcommand{\rC}{\mathrm{C}}
\newcommand{\rV}{\mathrm{V}}
\newcommand{\ls}{\mathrm{LSC}_{+}(X)}
\newcommand{\ce}{\mathrm{C}^{*}(X)}
\newcommand{\lsc}{\mathrm{LSC}}
\newcommand{\fB}{\mathfrak{B}}
\newcommand{\fM}{\mathfrak{M}}
\newcommand{\bt}{\begin{theorem}}
\newcommand{\et}{\end{theorem}}
\newcommand{\bl}{\begin{lemma}}
\newcommand{\el}{\end{lemma}}
\newcommand{\bc}{\begin{corollary}}
\newcommand{\ec}{\end{corollary}}
\newcommand{\bp}{\begin{proof}}
\newcommand{\ep}{\end{proof}}
%%%%%%%%%%%%%%%%%%%%%%%%%%%%
% دستورهایی برای سفارشی کردن سربرگ صفحات
\csname@twosidetrue\endcsname
\pagestyle{fancy}
\setlength{\headheight}{16pt}
\fancyhf{} 
\fancyhead[OL,EL]{\thepage}
\fancyhead[OR]{\small\rightmark}
\fancyhead[ER]{\small\leftmark}
\renewcommand{\chaptermark}[1]{%
\markboth{\thechapter.\ #1}{}}
%%%%%%%%%%%%%%%%%%%%%%%%%%%%%

% دستورهایی برای سفارشی کردن صفحات اول فصل‌ها
\makeatletter
\def\@makechapterhead#1{%
   \vspace*{-30\p@}% default: 50pt
   {\parindent \z@ %\raggedleft \normalfont% default: \raggedright
     \ifnum \c@secnumdepth >\m@ne
      \begin{flushleft}   
         \huge\bfseries \@chapapp\space \thechapter
         \par\nobreak
         \vskip 20\p@% default: 20\p@
     \fi
     \interlinepenalty\@M
     \Huge \bfseries #1\par\nobreak
     \vskip 120\p@
     \end{flushleft}
   }}
 \makeatother
