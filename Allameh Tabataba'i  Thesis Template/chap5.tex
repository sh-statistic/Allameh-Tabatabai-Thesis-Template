\chapter{
اختیار خرید آمریکایی 
}
\label{se:option}
%
در فصل‌های سوم و چهارم نشان داده شد که توابع پایه شعاعی از دقت بالایی برای تقریب مسایل چند بعدی برخوردار هستند. همچنین نشان داده شد که روش‌های مبتنی بر بدون شبکه‌بندی بصورت قابل توجهی حجم محاسبات را کاهش می‌دهد. بنابراین این‌گونه روش‌ها بعنوان روشی کارا برای حل مسایل مالی محسوب می‌شوند. تقریب عددی اختیار خریدهای اروپایی و آمریکایی در حالت یک بعدی، اولین بار توسط 
\citep{Hon,McLain}
مورد بررسی قرار گرفت. 
\citep{Pettersson}،
اختیار خریدهای چند بعدی را مورد تقریب عددی قرار دادند. اختیار خرید‌های اروپایی و آمریکایی در حالت یک و دو بعدی توسط 
\citep{Fasshauer}
مورد مطالعه قرار گرفت. در همه مطالعات فوق الذکر که برای حل عددی اختیار خریدها انجام شده، توابع پایه شعاعی در حالت موضعی به کار گرفته شده است. همانطور که در فصل قبل ذکر شد؛ توابع پایه شعاعی مبتنی بر افراز واحد یک روش موضعی است که منجربه تقریب‌های با دقت بالا و ماتریس‌های تنک می‌شود که مناسب برای مسایل چند بعدی است.

در این فصل، اختیار خریدهای چندبعدی را در نظر می‌گیریم و روش توابع پایه شعاعی مبتنی بر افراز واحد را برای بررسی عددی این نوع مسایل به کار می‌بریم.

