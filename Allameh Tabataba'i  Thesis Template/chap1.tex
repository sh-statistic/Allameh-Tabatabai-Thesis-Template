% 
\chapter{آشنایی و کلیات}


عموما رفتار پدیده‌های فیزیکی را می‌توان توسط معادلات دیفرانسیل با مشتقات جزیی مدل‌بندی کرد. 
\index{معادلات دیفرانسیل}
با ابعاد بالا نقش مهمی در بسیاری از علوم مهندسی و کاربردی بویژه مکانیک سیالات، ریاضیات مالی، فیزیک جامدات، فیریک شیمی و ... ایفا می‌کنند.
\section{مقدمه}
اخیرا، اختیار خریدها بدلیل کاربرد وسیع‌ در بازارهای مالی اهمیت زیادی کسب کرده‌اند. خرید قراردادهای مالی با دارایی‌های مورد معامله چندگانه بیش از پیش مورد توجه قرار گرفته است. این چنین اختیار خریدهایی با معادله چند بعدی بلک-شولز
\LTRfootnote{Black--Scholes}
و یا صورت تعمیم یافته آن مدل بندی می‌شود
\citep{Zvan,Kwok}.

حل عددی معادلات دیفرانسیل با مشتقات جزیی موضوع تحقیقاتی در زمینه‌های علمی متفاوت بوده است. روش‌های تفاضلات متناهی، روش‌های اجزاء متناهی
و روش‌های طیفی  
از جمله روش‌های متداول در ادبیات موضوع برای حل عددی این گونه مسایل می‌باشند. دقت روش‌های فوق الذکر تحت تاثیر شبکه‌بندی نقاط و گسسته سازی دامنه می‌باشد
\citep{Larsson}.

\section{پیشینه پژوهش}
نمونه ارجاع به مرجع فارسی 
\citep{Vahedi87}.
\subsection{زير بخشی از پیشینه پژوهش}

\subsubsection{زير بخشی از زير بخشی از پیشینه پژوهش}

\subsubsection{زير بخشی دوم از زير بخشی از پیشینه پژوهش}
در دو دهه اخیر، روش‌های هم‌محلی مبتنی بر توابع پایه شعاعی بدلیل کاربرد آن‌ها برای حل عددی معادلات دیفرانسیل و معادلات دیفرانسیل با مشتقات جزیی بسیار مورد توجه واقع شده‌اند
\citep{Platte,Wu,Larsson}.
روش‌های مبتنی بر توابع پایه شعاعی  برای فائق آمدن به ضعف‌های روش‌های متداول قبلی پدید آمدند. مهم‌ترین برتری این روش‌ها این است که نیازی به شبکه‌بندی دامنه نیست و برای تقریب جواب فقط به نقاط پراکنده از دامنه نیاز است. به همین دلیل این روش‌ها موسوم به روش‌های بدون شبکه‌بندی هستند. این روش‌ها به آسانی قابل اجرا بر روی مسایل با بعد بالا و یا با دامنه‌های پیچیده هستند. خواص جالب همراه با همگرایی خوب  (نرخ نمایی در برخی موارد) این روش‌ها را بعنوان روشی کارا برای حل معادلات با مشتقات جزیی معرفی می‌کند.

طی سالیان متمادی، توابع پایه شعاعی بعنوان درونیاب‌های نقاط پراکنده چند‌بعدی شناخته شده بودند
\citep{Wendland}.
نتایج عالی این درونیاب‌ها در نقاط پراکنده انگیزه‌ای برای توسیع آن‌ها برای حل عددی معادلات دیفرانسیلی شد. چند نمونه از روش‌های بدون شبکه‌بندی موجود و در حال مطالعه در ادبیات موضوع عبارتند از: هم‌محلی توابع پایه شعاعی نامتقارن
\citep{Larsson},
هم‌محلی توابع پایه شعاعی متقارن
\citep{Rieger,Hon}
و روش افراز واحد مبتنی بر توابع پایه شعاعی
\citep{McLain}.


\section{هدف پژوهش}
%%
%%
%%
در این پایان‌نامه، هدف بررسی روش‌های مبتنی بر توابع پایه شعاعی در حالت موضعی و فراگیر برای تقریب جواب معادلات دیفرانسیل با مشتقات جزیی از دیدگاه محاسباتی و تئوری است. هدف اولیه این تحقیق، توسیع هم‌محلی توابع پایه شعاعی فراگیر برای دسته وسیعی از معادلات دیفرانسیل با مشتقات جزیی وابسته به زمان با شرایط مرزی متفاوت می‌باشد. همچنین این روش برای حل عددی مسایل با مشتقات از مراتب بالاتر و وابسته به زمان مانند معادله روزنا 
مورد بررسی قرار گرفته است. علاوه بر این آنالیز خطای روش‌های ارایه شده  هنگامی که روش‌ ترازی
برای حالت گسسته شده به کار رفته، مورد تحلیل قرار گرفته است.

هدف بعدی در این پایان نامه،  توسیع هم‌محلی توابع پایه شعاعی در حالت موضعی مانند توابع پایه شعاعی مبتنی بر روش تفاضلات متناهی و توابع پایه شعاعی مبتنی بر روش افراز واحد برای حل عددی معادلات دیفرانسیل با مشتقات جزیی وابسته به زمان می‌باشد. نهایتا هدف ایجاد یک الگوریتم کارا برای بررسی عددی مسایل با مقیاس بالا و چند بعدی و پیاده سازی آن برای اختیار خریدهای چند بعدی است.
%%
\section{تعریف‌ها و مفهوم‌های پایه‌ای}
%%
\section{چشم‌انداز} 
%%
%%
این پایان‌نامه به صورت زیر طراحی و نگارش شده است.

در فصل
\ref{se:rbf}
توابع پایه شعاعی را بصورت خلاصه معرفی می‌شوند، بعلاوه خواص توابع پایه شعاعی متفاوت و کاربرد آن‌ها برای درونیابی چند بعدی مورد بحث و بررسی قرار گرفته است. همچنین درونیابی با این توابع از دیدگاه تئوری و محاسباتی بطور مختصر بیان شده است.

فصل
\ref{se:grbf}
کاربرد هم‌محلی توابع پایه شعاعی فراگیر برای معادلات دیفرانسیل با مشتقات جزئی  غیر خطی وابسته به زمان را مورد بحث قرار می‌دهد. روش نقطه تصوری
و روش بازتصویر
برای مسایل مقدار مرزی با مشتقات از مرتبه بالاتر معرفی شده است. آنالیز خطا و پایداری روش‌های ارایه شده، مورد بررسی قرار گرفته است.


فصل
\ref{se:rbfloc}
هم‌محلی توابع پایه شعاعی موضعی را معرفی می‌کند. جزئیات محاسباتی توابع پایه شعاعی موضعی مبتنی بر تفاضلات متناهی برای معادله شرودینگر
مطرح شده است. دقت محاسباتی این روش که منجر به ماتریس ضرایب تنک می‌شود با روش فراگیر و سایر روش‌ها مقایسه شده است. همچنین روش توابع پایه شعاعی بر اساس افراز واحد بعنوان یک روش کارای موضعی برای مسایل با مقیاس بالا معرفی شده است. پیاده سازی این روش برای مسایل وابسته به زمان ارایه شده و جزییات محاسباتی آن برای معادله نفوذ گرمای وابسته به زمان دوبعدی
نشان داده شده است.

%


