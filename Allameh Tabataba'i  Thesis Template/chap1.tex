% 
\chapter{کلیات پژوهش}\label{se:chapter1}

\section{مقدمه}
در دهه‌های اخیر، با پیشرفت فراوان فناوری اطلاعات و ارتباطات، نحوه‌ی تولید، جمع‌آوری، و مدیریت داده‌ها به یکی از چالش‌های اساسی در حوزه تکنولوژی تبدیل شده است. امروزه داده‌ها به عنوان یک دارایی ارزشمند شناخته می‌شوند و نقش بسیار مهمی در تصمیم‌گیری‌ها و توسعه‌ی فناوری ایفا می‌کنند. با این حال، این حجم عظیم از داده‌ها همراه با ویژگی‌های مختلف و بدون ساختار،چالش‌هایی در زمینه مدیریت، تحلیل، و بهره‌برداری از آنها ایجاد کرده است.
یکی از مسائل اساسی موجود در حوزه داده، وجود داده‌های بدون ساختار است. این داده‌ها ممکن است از منابع مختلفی چون رسانه‌های اجتماعی، وب، سنسورها و دستگاه‌های مختلف جمع‌آوری شوند. عدم ساختار مشخص و یکنواخت در این داده‌ها باعث ایجاد چالش‌هایی در تحلیل و استفاده از آنها می‌شود. از این رو، توسعه روش‌ها و الگوریتم‌های موثر برای استخراج اطلاعات مفید از داده‌های بدون ساختار از اهمیت بسیاری برخوردار است.
همچنین، با گسترش اینترنت اشیا، داده‌ها از منابع متنوعی چون حسگرها، دستگاه‌های هوشمند، و سیستم‌های مختلف جمع‌آوری می‌شوند. این حجم عظیم از داده‌ها، افزایش پیچیدگی تحلیل و مدیریت داده را به همراه داشته است. در این زمینه، امکان استفاده بهینه از داده‌های اینترنت اشیا‌ء و تحلیل صحیح آنها، نقش مهمی در پیشرفت فناوری اطلاعات و ارتباطات ایفا می‌کند.
در این پژوهش، ما به بررسی فرایندهای مختلف تبدیل داده‌های بدون ساختار به داده‌ساختار یافته و استفاده از روش‌های یادگیری ماشین به منظور کشف الگوها از داده‌ بدون ساختار اینترنت اشیاء می‌پردازیم.
\section{بیان مسئله}
درسال‌های اخیر اینترنت اشیاء $(IoT)$ بسیار مورد توجه پژوهش‌گران قرارگرفته‌ است. این فناوری به کمک حسگرهایی، ‌اطلاعات را از محیط اطراف خود دریافت می‌کند و با شبکه‌ها و دستگاه‌های دیگر ارتباط برقرار می‌کند و داده‌های گردآوری شده را به سیستم‌های مرکزی ارسال می‌کند. این فناوری یکی از ابزارهای مهم و قدرتمند درجهت ارتقای امنیت، کیفیت زندگی و بهره‌وری به‌کارگرفته می‌شود و در صنایع مختلفی چون سلامت و پزشکی، کشاورزی، خودروسازی، آب و هوا و غیره به‌عنوان ابزاری برای بهبود کارایی و کاهش هزینه‌ها استفاده می‌شود. با استفاده از اینترنت اشیاء می‌توان از این اطلاعات برای تحلیل و پیش‌بینی رفتارهایی که در آینده ممکن‌ است رخ‌ دهد، استفاده‌کرد و در نتیجه امکان پیش‌بینی مشکلات و راه‌حل‌هایی برای آن وجود ‌دارد.

یکی از چالش‌های موجود در زمینه اینترنت اشیاء پردازش داده‌ها و کاربرد‌های آن است. این چالش به دلیل وجود داده‌های بدون ساختار ایجاد‌شده از طریق حسگرها، داده‌های شبکه‌های اجتماعی و تارنما (وب) که توسط شرکت‌ها و سازمان‌ها گرد‌آوری شده‌اند، به وجود‌آمده است. با توجه به تعداد بالای داده‌های بدون ساختار موجود در حوزه‌های مختلف درصورتی که داده‌های بدون ساختار به داده‌های ساختاریافته تبدیل شوند این امکان را به ما می‌دهد که بتوانیم از آن‌ها به نحو بهتری استفاده کنیم، الگوهایی را که درآن‌ها وجود دارد شناسایی کنیم و برای پیش‌بینی رفتار آینده، تصمیم‌گیری‌های بهتری بگیریم.
\\
در این پژوهش، ما به بررسی فرایندهای مختلف تبدیل داده‌های بدون ساختار به داده‌ساختار یافته و استفاده از روش‌های یادگیری ماشین به منظور کشف الگوها از داده‌ بدون ساختار اینترنت اشیاء می‌پردازیم.
\section{پیشینه پژوهش}
با افزایش حجم داده‌ها و تنوع منابع تولید داده،چالش‌های زیادی در زمینه مدیریت و تجزیه و تحلیل داده‌های بزرگ بدون ساختار به‌وجود آمده‌است که پژوهشگران راهکارهایی برای برطرف کردن این چالش‌ها ارائه‌ کرده‌اند.برخی از پژوهش‌ها در این زمینه، به بررسی روش‌های موجود برای پردازش و مدیریت داده‌های بدون ساختار، تبدیل داده‌های بدون ساختار به داده‌های ساختاری و به‌دست آوردن اطلاعات معنی‌دار از داده‌ها می‌پردازند. 

\cite{abidin2010}
به ارائه تحقیقات در مورد شناسایی داده‌های بدون ساختار ، استخراج و طبقه‌بندی صفحات وب که سپس به سند ساختاریافته به زبان نشانه‌گذاری (XML) تبدیل می‌شود پرداخته که بعداً در یک پایگاه داده چندرسانه‌ای ذخیره می شود.

\citep{gandhi2016}.
پژوهشی با موضوع استخراج اطلاعات از منابع نا‌ساختار و نیمه‌ساختار در زمینه فناوری اطلاعات با استفاده از\lr{semantic Web}، انجام داده‌است. در این پژوهش، نویسنده با ارائه یک سیستم پیشنهادی، اطلاعات مفید و ساختاری موردنیاز از منابع مختلف مثل مجلات علمی معتبر در زمینه فناوری اطلاعات را استخراج کرده است. با استفاده از \lr{semantic Web}، داده‌ها به صورت ساختاری و با معنا استخراج شده و معنای داده‌ها در طول استخراج حفظ شده‌است. هدف این پژوهش، کاهش زمان و هزینه‌های مرتبط با خواندن مجلات به‌طور کامل بوده‌است که در نتیجه می‌تواند به عنوان یک راه‌حل کارآمد در استخراج اطلاعات از منابع نا‌ساختار و نیمه‌ساختار، به ویژه در زمینه فناوری اطلاعات، مورد استفاده قرارگیرد.

\cite{mishra2017}
به بررسی تکنیک‌هایی برای تجزیه وتحلیل داده‌های بدون ساختار برای استخراج اطلاعات معنی‌دار پنهان در داده‌های بزرگ پرداخته‌است.

\cite{sambrekar2018}
به ارائه یک روش پیشنهادی برای تبدیل داده‌های بدون ساختار کشاورزی به داده های نیمه‌ساختار‌یافته یا ساختار‌یافته با استفاده از Couchbase ابزار NOSQL ارائه شده است. Couchbase پایگاه داده سند با معماری توزیع شده‌است که مقیاس پذیری ، عملکرد و در دسترس بودن را فراهم می‌کند.

\cite{tekli2016}
در این مقاله، نویسنده بررسی کرده‌است که چگونه می توان از تحلیل و تمایز داده‌های نیمه ساختار یافته در قالب XML برای برنامه های هوشمند استفاده کرد. او به بررسی روش‌های مختلفی برای تحلیل و تمایز داده‌های XML و نیمه ساختار‌یافته پرداخته‌است و همچنین به بررسی کاربردهای مختلفی که از تحلیل داده‌های XML می‌توان داشت، از جمله خوشه‌بندی داده‌ها، یادگیری انطباقی و ... همچنین، نویسنده به بررسی چالش‌های مرتبط با تحلیل داده‌های XML پرداخته‌است.

\cite{artiles2004}
یک روش برای تشخیص معنای کلمات بر اساس شباهت میان کلمات در فضای متنی ارائه کرده‌است.

\cite{abdullah2013}
پژوهشی در مورد فرآیند نقشه برداری داده‌های بدون‌ساختار به داده‌های ساختاری انجام داده‌است در این مقاله چهار (4) فرآیند اصلی که شامل استخراج، طبقه بندی، توسعه مخازن و نقشه‌برداری داده‌ها با قصد کمک به تولید داده‌ها و اطلاعات جدید است که ساختار‌یافته‌تر، جامع‌تر، جمعی و متنوع‌تر در مباحث هستند تا نیازهای سازمانی را ارائه دهند. خروجی از این مطالعه اطمینان از فرآیند نقشه‌برداری داده‌های بدون‌ساختار به داده‌های ساخت‌یافته، داده‌های بدون ساختار را به عنوان دارایی‌های تجدید پذیر مفید، آگاه و معنادار برای خدمت به عملکردهای سازمانی نشان می‌دهد.

\cite{rusu2013}
به تبدیل داده‌های بدون‌ساختار و نیمه‌ساختار‌یافته به دانش پرداخته‌است. استخراج دانش، فرایند ایجاد دانش از داده‌های ساختاری، بدون‌ساختار و نیمه‌ساختار‌یافته است. این مقاله به بررسی امکانات استخراج دانش از داده‌های بدون‌ساختار و نیمه‌ساختار یافته می‌پردازد. نظریه‌ها و ابزارهای استخراج دانش در زمینه نوظهور، کشف دانش در پایگاه داده‌ها (KDD) مورد بررسی قرار‌گرفته و روند کلی KDD چند مرحله‌ای بیان شده است. درادامه، برنامه‌های اخیر KDD در دنیای واقعی به صورت خلاصه بیان‌شده و چالش‌هایی که برای تحقیقات و توسعه آینده در سیستم‌های KDD وجود دارد، بررسی شده است.

\section{هدف پژوهش}
%%
%%
%%
هدف از انجام این پژوهش آشنایی با روش‌های مختلف تبدیل داده‌های بدون ساختار به داده ساختار یافته و استفاده از الگوریتم‌های یادگیری ماشینی برای کشف الگو از داده‌های ناساختار حسگرهای اینترنت اشیاء می‌باشد.

\section{چشم‌انداز} 
%%
%%
فصل‌بندی مطالب این پایان‌نامه به صورت زیر طراحی و نگارش شده است.

فصل 
\ref{se:chapter1}
از کلیات این پژوهش و چیستی انجام آن برخوردار است؛ بدین گونه که به بیان مسئله و مرور ادبیات مربوط به آن و هدف از این پژوهش پرداخته شد.

 فصل۲
%\ref{se:rbf}
با هدف ارائه چارجوب و مبانی نظری این پژوهش جمع‌آوری خواهد‌ شد تا زمینه برای بررسی موضوع مورد مطالعه این پایان‌نامه فراهم شود. بدین‌ترتیب که در ابتدا با انواع دسته‌بندی داده‌ها آشنا خواهیم شد و سپس به بررسی الگوریتم‌های مختلف یادگیری ماشین می‌پردازیم. 

فصل۳
%\ref{se:grbf}
در این فصل به بررسی مبانی نظری روش‌های مختلف تبدیل داده‌های بدون ساختار به داده‌های ساختاریافته خواهیم‌پرداخت.


فصل۴
%\ref{se:rbfloc}
که آخرین فصل پایان‌نامه می‌باشد، به پیاده‌سازی یک روش مطرح‌شده در فصل \ref{se:grbf}، بر روی داده‌ بدون‌ساختار اینترنت اشیاء خواهد‌شد.بدین گونه که از طریق مجموعه داده‌های استانداردی که از پایگاه‌های داده‌ی معتبر استخراج‌ شده‌است،این الگوریتم‌ها را پیاده‌سازی خواهیم‌کرد و نتایج خروجی از این پیاده‌سازی‌ها مورد تجزیه و تحلیل قرار‌ خواهد گرفت و در نهایت میزان دقت آن‌ها مورد بحث قرار ‌می‌گیرد.


 
