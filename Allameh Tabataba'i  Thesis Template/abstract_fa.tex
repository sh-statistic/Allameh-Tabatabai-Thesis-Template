%
%
%														پس از اضافه نمودن خلاصه فارسی، این فایل را ذخیره نموده و سپس فایل اصلی رساله را اجرا کنید
%
%
\thispagestyle{empty}
\noindent
\centerline{\textbf{\large{چکیده}}} \\
در این پایان‌نامه، روش هم‌محلی مبتنی بر توابع پایه شعاعی فراگیر برای تقریب جواب دستگاه‌ معادلات دیفرانسیل غیر خطی وابسته به زمان بررسی می‌شود. تا کنون پیاده سازی شرایط مرزی چندگانه با استفاده از این روش برای مسایل وابسته به زمان حتی در حالت یک بعدی نیز مورد بررسی قرار نگرفته است. ما پیاده سازی شرایط مرزی چندگانه با روش‌های مبتنی بر توابع پایه شعاعی را از دیدگاه تئوری و عملی مورد بحث و بررسی قرار داده‌ایم. از لحاظ تئوری، آنالیز خطای روش توابع پایه شعاعی فراگیر برای معادلات دیفرانسیل غیرخطی با مشتقات جزیی مورد بررسی قرار گرفته است. از لحاظ محاسباتی، جزییات پیاده سازی برای روش‌های ارایه شده، بحث شده است. همچنین کارایی آن‌ها با مقایسه‌ی جواب دقیق و روش‌های دیگر  نشان داده شده است. دو روش هم‌محلی مبتنی بر کاربرد موضعی توابع پایه شعاعی برای تقریب جواب معادلات دیفرانسیل با مشتقات جزیی وابسته به زمان مورد بحث قرار گرفته است. توابع پایه شعاعی مبتنی بر روش افراز واحد بعنوان یک روش کارا برای حل این نوع مسایل معرفی شده است. خصوصیات روش افراز واحد کمک می‌کند تا از انعطاف پذیری بیشتری برای انتخاب نقاط موضعی هنگام تقریب جواب برخوردار شویم. در ادامه ماتریس متناظر با عملگرهای مشتق را از این روش استخراج می‌کنیم و نشان می‌دهیم که این چنین ماتریس‌هایی تنک بوده و کارایی بهتری برای حل مسایل  چندبعدی دارد. نهایتا، بعنوان یک کاربرد عملی، این روش برای تقریب جواب مساله اختیار خرید دوبعدی آمریکایی پیاده سازی شده است.
% \newpage
% \thispagestyle{empty}
% \noindent
% در صورت نياز به صفحه دوم چكيده سه سطر فوق را فعال كنيد و متن چكيده را در اين قسمت تايپ كنيد.
\\

\noindent
\textbf{واژگان کلیدی:} 
\emph
{معادلات دیفرانسیل با مشتقات جزیی، توابع پایه شعاعی، روش افراز واحد، روش هم‌محلی، آنالیز خطا.}