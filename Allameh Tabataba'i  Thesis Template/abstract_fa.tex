%
%
%														پس از اضافه نمودن خلاصه فارسی، این فایل را ذخیره نموده و سپس فایل اصلی رساله را اجرا کنید
%
%
\thispagestyle{empty}
\noindent
\centerline{\textbf{\large{چکیده}}} \\
درسال‌های اخیر اینترنت اشیاء $(IoT)$ بسیار مورد توجه پژوهش‌گران قرارگرفته‌ است. این فناوری به کمک حسگرهایی، ‌اطلاعات را از محیط اطراف خود دریافت می‌کند و با شبکه‌ها و دستگاه‌های دیگر ارتباط برقرار می‌کند و داده‌های گردآوری شده را به سیستم‌های مرکزی ارسال می‌کند. این فناوری یکی از ابزارهای مهم و قدرتمند درجهت ارتقای امنیت، کیفیت زندگی و بهره‌وری به‌کارگرفته می‌شود و در صنایع مختلفی چون سلامت و پزشکی، کشاورزی، خودروسازی، آب و هوا و غیره به‌عنوان ابزاری برای بهبود کارایی و کاهش هزینه‌ها استفاده می‌شود. با استفاده از اینترنت اشیاء می‌توان از این اطلاعات برای تحلیل و پیش‌بینی رفتارهایی که در آینده ممکن‌ است رخ‌ دهد، استفاده‌کرد و در نتیجه امکان پیش‌بینی مشکلات و راه‌حل‌هایی برای آن وجود ‌دارد.

یکی از چالش‌های موجود در زمینه اینترنت اشیاء پردازش داده‌ها و کاربرد‌های آن است. این چالش به دلیل وجود داده‌های بدون ساختار ایجاد‌شده از طریق حسگرها، داده‌های شبکه‌های اجتماعی و تارنما (وب) که توسط شرکت‌ها و سازمان‌ها گرد‌آوری شده‌اند، به وجود‌آمده است. با توجه به تعداد بالای داده‌های بدون ساختار موجود در حوزه‌های مختلف درصورتی که داده‌های بدون ساختار به داده‌های ساختاریافته تبدیل شوند این امکان را به ما می‌دهد که بتوانیم از آن‌ها به نحو بهتری استفاده کنیم، الگوهایی را که درآن‌ها وجود دارد شناسایی کنیم و برای پیش‌بینی رفتار آینده، تصمیم‌گیری‌های بهتری بگیریم.
\\
در این پایان‌نامه، ما به بررسی فرایندهای مختلف تبدیل داده‌های بدون ساختار به داده‌ساختار یافته و استفاده از روش‌های یادگیری ماشین به منظور کشف الگوها از داده‌ بدون ساختار اینترنت اشیاء می‌پردازیم.
% \newpage
% \thispagestyle{empty}
% \noindent
% در صورت نياز به صفحه دوم چكيده سه سطر فوق را فعال كنيد و متن چكيده را در اين قسمت تايپ كنيد.
\\

\noindent
\textbf{واژگان کلیدی:} 
\emph
{الگوریتم‌های یادگیری ماشین، داده‌های بدون ساختار، داده‌های ساختاریافته، رده‌بندی، حسگرهای‌ اینترنت اشیاء.}