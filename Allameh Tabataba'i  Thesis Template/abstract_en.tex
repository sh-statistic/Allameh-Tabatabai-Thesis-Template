%
%
%														پس از اضافه نمودن خلاصه انگلیسی، این فایل را ذخیره نموده و سپس فایل اصلی رساله را اجرا کنید
%
%
\thispagestyle{empty} 	%		prevents tex from numbering of this page
\begin{latin} 					% 	xepersian enviorment

\centerline{\textbf{\large{Abstract}}}
\vskip 1cm
\noindent 
In recent years, the Internet of Things (IoT) has received much attention from researchers. This technology receives information from its surroundings with the help of sensors and communicates with networks and other devices and sends the collected data to central systems. This technology is one of the important and powerful tools used to improve security, quality of life and productivity, and it is used in various industries such as health and medicine, agriculture, automotive, weather, etc. as a tool to improve efficiency and reduce costs. By using the Internet of Things, this information can be used to analyze and predict behaviors that may occur in the future, and as a result, it is possible to predict problems and solutions for them.

One of the challenges in the field of Internet of Things is data processing and its applications. This challenge has arisen due to the existence of unstructured data created through sensors, social network and web data collected by companies and organizations. Considering the high number of unstructured data available in different fields, if the unstructured data is converted into structured data, it gives us the possibility to use them in a better way, to identify the patterns that exist in them and to predict the behavior. Let's make better decisions in the future.
\\ In this thesis, we investigate various processes of converting unstructured data into structured data and using machine learning methods to discover patterns from unstructured data of the Internet of Things.\\

\noindent
\textbf{Keywords:} 
\emph{Classification, IoT Sensors, Machine Learning Algorithm, Structed Data, Unstructed Data.}
\end{latin}