%
%
%														پس از اضافه نمودن خلاصه انگلیسی، این فایل را ذخیره نموده و سپس فایل اصلی رساله را اجرا کنید
%
%
\thispagestyle{empty} 	%		prevents tex from numbering of this page
\begin{latin} 					% 	xepersian enviorment
	
	\centerline{\textbf{\large{Abstract}}}
	\vskip 1cm
	\noindent 
	In this thesis, we invistigate the global RBF collocation method
	for the non-linear system of the time-dependent PDEs. Even for the
	one-dimensional case, how to implement multiple boundary
	conditions for a time-dependent problem is not obvious. We explore
	this difficulty and possible remedies at theoretical and practical
	levels. Theoretically, we study the error estimate of global RBF
	collocation method when the method of line is applied  for
	discretized form of the time-dependent non-linear PDEs.
	Computationally, we present the details of implementation for the
	new proposed methods. The efficiency of the proposed methods is
	demonstrated by comparing with the other methods.
	We also study the local RBF collocation methods for numerical
	solution of the PDEs. The partition of unity based on RBF is
	introduced as a robust method for the time-dependent PDEs. The
	properties of the partition of unity method (PUM) allow to have
	greater flexibility to adjust local approximation. We generate the
	differentiation matrices based on RBF--PUM discretizations and
	show that such schemes can lead to sparse matrices that is
	suitable for numerical study of the large scale problems. 
	As a more practical approach, we develop the RBF--PUM
	for the numerical approximation of the two-dimensional American
	put option pricing.\\
	
	\noindent
	\textbf{Keywords:} 
	\emph{PDEs, Radial basis function, Partition of unity method, collocation method, Error estimate.}
\end{latin}