
\chapter{نتيجه‌گيری و پیشنهادهای آتی}
\label{se:Conclusion}

در این پایان‌نامه توابع پایه شعاعی بعنوان یک روش کارا برای حل عددی معادلات دیفرانسیل با مشتقات جزیی وابسته به زمان مورد استفاده قرار گرفته‌اند. نتایج بدست آمده را می‌توان بصورت زیر بیان کرد:
\begin{enumerate}
\item 
روش هم‌محلی مبتنی بر توابع پایه شعاعی فراگیر منجر به تقریب با دقت بهتری در مقایسه با روش‌های کرنک-نیکلسون و روش عناصر متناهی می‌باشد. 
\item 
فصل پنجم این رساله پیاده سازی روش توابع پایه شعاعی مبتنی بر روش افراز واحد برای تقریب جواب اختیار خرید دوبعدی آمریکایی را مورد بررسی قرار داده و جزییات محاسباتی آن را بیان می‌کند.
\end{enumerate}
به عنوان پیشنهاد برای کارهای آینده به موارد زیر اشاره می‌کنیم:
\begin{enumerate}
\item
تعمیم روش توابع پایه شعاعی فراگیر به مسایل معکوس.
\item
بررسی پایداری روش‌های توابع پایه شعاعی مبتنی بر افراز واحد و تفاضلات متناهی.
\item
ایجاد تابعی برای حل عددی معادلات دیفرانسیل با مشتقات جزیی وابسته به زمان با استفاده از توابع پایه شعاعی.
\end{enumerate}